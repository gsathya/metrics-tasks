\documentclass{article}
\begin{document}
\author{George Danezis\\{\tt gdane@microsoft.com}}
\title{An anomaly-based censorship-detection\\system for Tor}
\date{August 11, 2011}
\maketitle

\section{Introduction}

The Tor project is currently the most widely used anonymity and censorship
resistance system worldwide.
As a result, national governments occasionally or regularly block access
to its facilities for replaying traffic.
Major blocking might be easy to detect, but blocking from smaller
jurisdictions, with fewer users, could take some time to detect.
Yet, early detection may be key to deploy countermeasures.
We have designed an ``early warning'' systems that looks for anomalies in
the volumes of connections from users in different jurisdictions and flags
potential censorship events.
Special care has been taken to ensure the detector is robust to
manipulations and noise that could be used to block without raising an
alert.

The detector works on aggregate number of users connecting to a fraction
of directory servers per day.
That set of statistics are gathered and provided by the Tor project in a
sanitised form to minimise the potential for harm to active users.
The data collection has been historically patchy, introducing wild
variations over time that is not due to censorship.
The detector is based on a simple model of the number of users per day per
jurisdictions.
That model is used to assess whether the number of users we observe is
typical, too high or too low.
In a nutshell the prediction on any day is based on activity of previous
days locally as well as worldwide.

\section{The model intuition}

The detector is based on a model of the number of connection from every
jurisdiction based on the number of connections in the past as well as a
model of ``natural'' variation or evolution of the number of connections.
More concretely, consider that at time $t_i$ we have observed $C_{ij}$
connections from country $j$.
Since we are concerned with abnormal increases or falls in the volume of
connections we compare this with the number of connections we observed at
a past time $t_{i-1}$ denoted as $C_{(t-1)j}$ from the same country $j$.
The ratio $R_{ij} = C_{ij} / C_{(i-1)j}$ summarises the change in the
number of users.
Inferring whether the ratio $R_{ij}$ is within an expected or unexpected
range allows us to detect potential censorship events. 

We consider that a ratio $R_{ij}$ within a jurisdiction $j$ is ``normal''
if it follows the trends we are observing in other jurisdictions.
Therefore for every time $t_i$ we use the ratios $R_{ij}$ of many
countries to understand the global trends of usage of Tor, and we compare
specific countries' ratios to this model.
If they are broadly within the global trends we assume no censorship is
taking place, otherwise we raise an alarm.

\section{The model details}

We model each data point $C_{ij}$ of the number of users connected at time
$t_i$ from country $j$ as a single sample of a Poisson process with a rate
$\lambda_{ij}$ modelling the underlying number of users.
The Poisson process allows us to take into account that in jurisdictions
with very few users we will naturally have some days of relatively low or
high usage---just because a handful of users may or may not use Tor in a
day.
Even observing zero users from such jurisdictions on some days may not be
a significant event. 

We are trying to detect normal or abnormal changes in the rate of change
of the rate $\lambda_{ij}$ between time $t_i$ and a previous time
$t_{i-1}$ for jurisdiction $j$ compared with other jurisdictions.
This is $\lambda_{ij} / \lambda_{(i-1)j}$ which for jurisdictions with a
high number of users is very close to $C_{ij} / C_{(i-1)j} = R_{ij}$.
We model $R_{ij}$'s from all jurisdictions as following a Normal
distribution $N(m,v)$ with a certain mean ($m$) and variance ($v$) to be
inferred.
This is of course a modelling assumption.
We use a normal distribution because given its parameters it represents
the distribution with most uncertainty: as a result the model has higher
variance than the real world, ensuring that it gives fewer false alarms of
censorship.

The parameters of $N(m,v)$ are inferred directly as point estimates from
the readings in a set of jurisdictions.
Then the probability of a given country ratio $R_{ij}$ is compared with
that distribution: an alarm is raised if the probability of the ratio is
above or below a certain threshold.

\section{The model robustness}

At every stage of detections we follow special steps to ensure the
detection is robust to manipulation by jurisdictions interested in
censoring fast without being detected.
First the parameter estimation for $N(m,v)$ is hardened: we only use the
largest jurisdictions to model ratios and within those we remove any
outliers that fall outside four inter-quartile ranges off the median.
This ensures that a jurisdiction with a very high or very low ratio does
not influence the model of ratios (and can be subsequently detected as
abnormal).

Since we chose jurisdictions with many users to build the model of ratios,
we can approximate the rates $\lambda_{ij}$ by the actual observed number
of users $C_{ij}$.
On the other hand when we try to detect whether a jurisdiction has a
typical rate we cannot make this assumption.
The rate of a Poisson variable $\lambda_{ij}$ can be inferred by a single
sample $C_{ij}$ using a Gamma prior, in which case it follows a Gamma
distribution.
In practice (because we are using a single sample) this in turn can be
approximated using a Poisson distribution with parameter $C_{ij}$.
Using this observation we extract a range of possible rates for each
jurisdiction based on $C_{ij}$, namely $\lambda_{ij_{min}}$ and
$\lambda_{ij_{max}}$.
Then we test whether that full range is within the typical range
distribution---if not we raise an alarm.

\section{The parameters}

The deployed model considers a time interval of seven (7) days to model
connection rates (i.e. $t_i$ - $t_{i-1} = 7$ days).
The key reason for a weekly model is our observation that some
jurisdictions exhibit weekly patterns.
A previous day model would then raise alarms every time weekly patterns
emerged.
We use the 50 largest jurisdictions to build our models of typical ratios
of traffic over time---as expected most of them are in countries where no
mass censorship has been reported.
This strengthens the model as describing ``normal'' Tor connection
patterns.

We consider that a ratio of connections is typical if it falls within the
99.99~\% percentile of the Normal distribution $N(m,v)$ modelling ratios.
This ensures that the expected rate of false alarms is about $1 / 10000$,
and therefore only a handful a week (given the large number of
jurisdictions).
Similarly, we infer the range of the rate of usage from each jurisdiction
(given $C_{ij}$) to be the 99.99~\% percentile range of a Poisson
distribution with parameter $C_{ij}$.
This full range must be within the typical range of ratios to avoid
raising an alarm.

\section{Further work}

The detector uses time series of user connections to directory servers to
detect censorship.
Any censorship method that does not influence these numbers would as a
result not be detected.
This includes active attacks: a censor could substitute genuine requests
with requests from adversary controlled machines to keep numbers within
the typical ranges.

A better model, making use of multiple previous readings, may improve the
accuracy of detection.
In particular, when a censorship event occurs there is a structural
change, and a model based on modelling the future on user loads before the
event will fail.
This is not a critical problem, as these ``false positives'' are
concentrated after real censorship events, but the effect may be confusing
to a reader.
On the other hand, a jurisdiction can still censor by limiting the rate of
censorship to be within the typical range for the time period concerned.
Therefore adapting the detector to run on longer periods would be
necessary to detect such attacks.

\end{document}

