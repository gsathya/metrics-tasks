\documentclass{article}
\usepackage{url}
\usepackage[pdftex]{graphicx}
\usepackage{graphics}
\usepackage{color}
\begin{document}
\title{Investigating scaling points to handle more bridges}
\author{Karsten Loesing\\{\tt karsten@torproject.org}}

\maketitle

\section{Introduction}

The current bridge infrastructure relies on a central bridge authority to
collect, distribute, and publish bridge relay descriptors.
We believe the current infrastructure can handle up to 10,000 bridges.

The scaling points involve the database of descriptors, the metrics portal
and its ability to handle this many descriptors for analysis, and the
reachability testing part of the code for the bridge authority.
We should investigate scaling points to handle more than 10,000 bridge
descriptors.

\section{Early results}

We started this analysis by writing a small tool to generate sample data
for BridgeDB and metrics-db.
This tool takes the contents from one of Tonga's bridge tarball as input,
copies them a given number of times, and overwrites the first two bytes of
relay fingerprints in every copy with 0000, 0001, etc.
The tool also fixes references between network statuses, server
descriptors, and extra-info descriptors.
This is sufficient to trick BridgeDB and metrics-db into thinking that
relays in the copies are distinct relays.
We used the tool to generate tarballs with 2, 4, 8, 16, 32, and 64 times
as many bridge descriptors in them.

In the next step we fed the tarballs into BridgeDB and metrics-db.
BridgeDB reads the network statuses and server descriptors from the latest
tarball and writes them to a local database.
metrics-db sanitizes two half-hourly created tarballs every hour,
establishes an internal mapping between descriptors, and writes sanitized
descriptors with fixed references to disk.

Figure~\ref{fig:bridgescaling} shows the results.

\begin{figure}[t]
\includegraphics[width=\textwidth]{bridge-scaling.png}
%\caption{}
\label{fig:bridgescaling}
\end{figure}

The upper graph shows how the tarballs grow in size with more bridge
descriptors in them.
This growth is, unsurprisingly, linear.
One thing to keep in mind here is that bandwidth and storage requirements
to the hosts transferring and storing bridge tarballs are growing with the
tarballs.
We'll want to pay extra attention to disk space running out on those
hosts.

The middle graph shows how long BridgeDB takes to load descriptors from a
tarball.
This graph is linear, too, which indicates that BridgeDB can handle an
increase in the number of bridges pretty well.
One thing we couldn't check is whether BridgeDB's ability to serve client
requests is in any way affected during the descriptor import.
We assume it'll be fine.
We should ask Aaron, if there are other things in BridgeDB that we
overlooked that may not scale.

The lower graph shows how metrics-db can or cannot handle more bridges.
The growth is slightly worse than linear.
In any case, the absolute time required to handle 25K bridges is worrisome
(we didn't try 50K).
metrics-db runs in an hourly cronjob, and if that cronjob doesn't finish
within 1 hour, we cannot start the next run and will be missing some data.
We might have to sanitize bridge descriptors in a different thread or
process than the one that fetches all the other metrics data.
We can also look into other Java libraries to handle .gz-compressed files
that are faster than the one we're using.
So, we can probably handle 25K bridges somehow, and maybe even 50K.
Somehow.

Finally, note that we left out the most important part of this analysis:
can Tonga, or more generally, a single bridge authority handle this
increase in bridges?
We're not sure how to test such a setting, or at least without running 50K
bridges in a private network.
We could imagine this requires some more sophisticated sample data
generation including getting the crypto right and then talking to Tonga's
DirPort.
If there's an easy way to test this, we'll do it.
If not, we can always hope for the best.
What can go wrong.

\section{Work left to do}

If we end up with way too many bridges, here are a few things we'll want
to look at updating:

\begin{itemize}
\item Tonga still does a reachability test on each bridge every 21 minutes
or so.
Eventually the number of TLS handshakes it's doing will overwhelm its cpu.
\item The tarballs we make every half hour have substantial overlap.
If we have tens of thousands of descriptors, we would want to get smarter
at sending diffs over to bridgedb.
\item Somebody should check whether BridgeDB's interaction with users
freezes while it's reading a new set of data.
\end{itemize}

%\bibliography{bridge-scaling}
%\bibliographystyle{plain}

\end{document}

