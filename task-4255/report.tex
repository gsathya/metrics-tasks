\documentclass{article}
\usepackage{url}
\usepackage[pdftex]{graphicx}
\usepackage{graphics}
\usepackage{color}
\begin{document}
\title{An Analysis of Tor Bridge Stability\\
{\large --- Making BridgeDB give out at least one stable bridge per
user ---}}
\author{Karsten Loesing\\{\tt karsten@torproject.org}}

\maketitle

\section{Introducing the instable bridges problem}

As of October 2011, the Tor network consists of a few hundred thousand
clients, 2\,400 public relays, and about 600 non-public bridge relays.
Bridge relays (in the following: bridges) are entry points which are not
publicly listed to prevent censors from blocking access to the Tor
network.
Censored users request a small number of typically three bridge addresses
from the BridgeDB service via email or http and then connect to the Tor
network only via these bridges.
If all bridges that a user knows about suddenly stop working, the user
needs to request a new set of bridge addresses from BridgeDB.
However, BridgeDB memorizes the user's email or IP address and only gives
out new bridges every 24 hours to slow down enumeration attempts.
The result is that a user who is unlucky enough to receive only unreliable
bridges from BridgeDB won't be able to connect to the Tor network for up
to 24 hours before requesting a new set of bridges.

In this report we propose that BridgeDB keeps bridge stability records,
similar to how the directory authorities keep relay stability records, and
includes at least one stable bridge in its responses to users.
In fact, BridgeDB currently attempts to do this by including at least one
bridge with the Stable flag assigned by the bridge authority in its
results.
This approach is broken for two reasons:
The first reason is that the algorithm that the bridge authority uses to
assign the Stable flag is broken to the extent that almost every bridge
has the Stable flag assigned.
The second reason is that the Stable flag was designed for clients to
select relays for long-running streams, not for BridgeDB to select
reliable entry points into the Tor network.
A better metric for stable bridges would be based on bridge uptime and on
the frequency of IP address changes.
We propose such a metric and evaluate its effectiveness for selecting
stable bridges based on archived bridge directories.

\section{Defining a new bridge stability metric}
\label{sec:defining}

The directory authorities implement a few relay stability metrics to
decide which of the relays to assign the Guard and Stable
flag~\cite{dirspec, loesing2011analysis}.
The requirements for stable bridges that we propose here are similar to
the entry guard requirements.
That is, stable bridges should have a higher fractional uptime than
non-stable ones.
Further, a stable bridge should be available under the same IP address and
TCP port.
Otherwise, bridge users who only know a bridge address won't be able to
connect to the bridge once it changes its address or port.
We propose the following requirements for a bridge to be considered
stable in the style of the Guard and Stable flag definition:

\label{def:bridgestability}
\begin{quote}
A bridge is considered stable if its \emph{Weighted Mean Time Between
Address Change} is at least the median for known active bridges or at
least 30~days, if it is `familiar', and if its \emph{Weighted Fractional
Uptime} is at least the median for `familiar' active bridges or at least
98~\%.
A bridge is `familiar' if 1/8 of all active bridges have appeared more
recently than it, or if it has been around for a \emph{Weighted Time} of
8~days.
\end{quote}

This bridge stability definition contains three main requirements:

\begin{itemize}
\item The \emph{Weighted Mean Time Between Address Change (WMTBAC)}
metric is used to track the time that a bridge typically uses the same IP
address and TCP port.
The (unweighted) MTBAC measures the average time between last using
address and port $a_0$ to last using address and port $a_1$.
This metric is weighted to put more emphasis on recent events than on past
events.
Therefore, past address sessions are discounted by factor $0.95$ every
12~hours.
The current session is not discounted, so that a WMTBAC value of 30 days
can be reached after 30 days at the earliest.
\item The \emph{Weighted Fractional Uptime (WFU)} metric measures the
fraction of bridge uptime in the past.
Similar to WMTBAC, WFU values are discounted by factor $0.95$ every
12~hours, but in this case including the current uptime session.
\item The \emph{Weighted Time (WT)} metric is used to calculate a bridge's
WFU and to decide whether a bridge is around long enough to be considered
`familiar.'
WT is discounted similar to WMTBAC and WFU, so that a WT of 8 days can be
reached after around 16 days at the earliest.
\end{itemize}

All three requirements consist of a dynamic part that depends on the
stability of other bridges (e.g., ``A bridge is familiar if 1/8 of all
active bridges have appeared more recently than it, \ldots'') and a static
part that is independent of other bridges (e.g., ``\ldots or if it has
been around for a Weighted Time of 8 days.'').
The dynamic parts ensure that a certain fraction of bridges is considered
stable even in a rather instable network.
The static parts ensures that rather stable bridges are not excluded even
when most other bridges in the network are stable.

\section{Extending BridgeDB to track bridge stability}

There are at least two code bases that could be extended to track bridge
stability and include at least one stable bridge in BridgeDB results: the
bridge authority and BridgeDB.
The decision for extending either code base affects the available data
for tracking bridge stability and is therefore discussed here.

The bridge authority maintains a list of all active bridges.
Bridges register at the bridge authority when joining the network, and the
bridge authority periodically performs reachability tests to confirm that
a bridge is still active.
The bridge authority takes snapshots of the list of active bridges every
30~minutes and copies these snapshots to BridgeDB.
BridgeDB parses these half-hourly snapshots and gives out bridges to users
based on the most recently known snapshot.

The bridge stability history can be implemented either in the bridge
authority code or in BridgeDB.
On the one hand, an implementation in BridgeDB has the disadvantage that
bridge reachability data has a resolution of 30 minutes whereas the bridge
authority would learn about bridges joining or leaving the network
immediately.
On the other hand, the bridge stability information is not used by
anything in the Tor software, but only by BridgeDB.
Implementing this feature in BridgeDB makes more sense from a software
architecture point of view.
In the following we assume that BridgeDB will track bridge stability based
on half-hourly snapshots of active bridge lists, the bridge network
statuses.

\section{Simulating bridge stability using archived data}

We can analyze how BridgeDB would track bridge stability and give out
stable bridges by using archived bridge descriptors.
These archives contain the same descriptors that BridgeDB uses, but they
are public and don't contain any IP addresses or sensitive pieces of
information.
In Section~\ref{sec:missingdata} we look at the problem of missing data
due to either the bridge authority or BridgeDB failing and at the effect
on tracking bridge stability.
We then touch the topic of how bridge descriptors are sanitized and how we
can glue them back together for our analysis in
Section~\ref{sec:sanitizing}.
Next, we examine typical bridge stability values as requirements for
considering a bridge as stable in Section~\ref{sec:requirements}.
In Section~\ref{sec:fractions} we estimate what fraction of bridges would
be considered as stable depending on the chosen stability requirements.
Finally, in Section~\ref{sec:selectedstability} we evaluate how effective
different requirement combinations are for selecting stable bridges.
Result metrics are how soon selected bridges change their address or what
fractional uptime selected bridges have in the future.

\subsection{Handling missing bridge status data}
\label{sec:missingdata}

The bridge status data that we use in this analysis and that would also be
used by BridgeDB to track bridge stability is generated by the bridge
authority and copied over to BridgeDB every 30~minutes.
Figure~\ref{fig:runningbridge} shows the number of running bridges
contained in these snapshots from July 2010 to June 2011.

\begin{figure}[t]
\includegraphics[width=\textwidth]{runningbridge.pdf}
\caption{Median number of running bridges as reported by the bridge
authority}
\label{fig:runningbridge}
\end{figure}

For most of the time the number of bridges is relatively stable.
But there are at least two irregularities, one in July 2010 and another
one in February 2011, resulting from problems with the bridge authority or
the data transfer to the BridgeDB host.
Figure~\ref{fig:runningbridge-detail} shows these two intervals in more
detail.

\begin{figure}[t]
\includegraphics[width=\textwidth]{runningbridge-detail.pdf}
\caption{Number of Running bridges during phases when either the bridge
authority or the BridgeDB host were broken}
\label{fig:runningbridge-detail}
\end{figure}

The missing data from July 14 to 27, 2010 comes from BridgeDB host not
accepting new descriptors from the bridge authority because of an
operating system upgrade of the BridgeDB host.
During this time, the bridge authority continued to work, but BridgeDB was
unable to learn about new bridge descriptors from it.

During the time from January 31 to February 16, 2011, the \verb+tor+
process running the bridge authority silently died, but the script to copy
descriptors to BridgeDB kept running.
In this case, BridgeDB received fresh tarballs containing stale
descriptors with a constant number of 687 relays, visualized in light
gray.
These stale descriptors have been excluded from the sanitized descriptors
and the subsequent analysis.
The bridge authority was restarted on February 16, 2011, resulting in the
number of running bridges slowly stabilizing throughout the day.

Both this analysis and a later implementation in BridgeDB need to take
extended phases of missing or stale data into account.

\subsection{Detecting address changes in sanitized descriptors}
\label{sec:sanitizing}

The bridge descriptor archives that we use in this analysis have been
sanitized to remove all addresses and otherwise sensitive
parts~\cite{loesing2011overview}.
Part of this sanitizing process is that bridge IP addresses are replaced
with keyed hashes using a fresh key every month.
More precisely, every bridge IP address is replaced with the private IP
address \verb+10.x.x.x+ with \verb+x.x.x+ being the 3 most significant
bytes of \verb+SHA-256(IP address | bridge identity | secret)+.

A side-effect of this sanitizing step is that a bridge's sanitized IP
address changes at least once per month, even if the bridge's real IP
address stays the same.
We need to detect these artificial address changes and distinguish them
from real IP address changes.

In this analysis we use a simple heuristic to distinguish between real IP
address changes and artifacts from the sanitizing process:
Whenever we find that a bridge has changed its IP address from one month
to the next, we look up how long both IP addresses were in use in either
month.
If both addresses were contained in bridge descriptors that were published
at least 36~hours apart, we consider them stable IP addresses and
attribute the apparent IP address change to the sanitizing process.
Otherwise, we assume the bridge has really changed its IP address.
Obviously, this simple heuristic might lead us to false conclusions in
some cases.
But it helps us handle cases when bridges rarely or never change their IP
address which would otherwise suffer from monthly address changes in this
analysis.

\subsection{Examining typical stability metric values}
\label{sec:requirements}

The definition of bridge stability on page~\pageref{sec:defining} contains
three different metrics, each of which having a dynamic and a static part.
The dynamic parts compares the value of a bridge's stability metric to the
whole set of running bridges.
Only those bridges are considered as stable that exceed the median value
(or the 12.5th percentile) of all running bridges.
The static requirement parts are fixed values for all stability metrics
that don't rely on the stability of other bridges.

Figure~\ref{fig:requirements} visualizes the dynamic (solid lines) and
static parts (dashed lines) of all three requirements.
The dynamic WMTBAC requirements are higher than previously expected.
A value of 60 means that, on average, bridges keep their IP address and
port for 60 days.
The dynamic values are cut off at 30 days by the static requirement which
should be a high enough value.
The goal here is to give blocked users a stable enough set of bridges so
that they don't have to wait another 24~hours before receiving new ones.

We can further see that the dynamic requirements are relatively stable
over time except for the two phases of missing bridge status data.
The first phase in July 2010 mostly affects WT, but neither WMTBAC nor
WFU.
The second phase in February 2011 affects all three metrics.
We can expect the selection of stable bridges during February 2010 to be
more random than at other times.

\begin{figure}[t]
\includegraphics[width=\textwidth]{requirements.pdf}
\caption{Dynamic requirements for considering a bridge as stable}
\label{fig:requirements}
\end{figure}

\subsection{Estimating fractions of bridges considered as stable}
\label{sec:fractions}

Requiring a bridge to meet or exceed either or both WMTBF or WFU metric
results in considering only a subset of all bridges as stable.
The first result of this analysis is to outline what fraction of bridges
would be considered as stable if BridgeDB used either or both
requirements.
In theory, all parameters in the bridge stability definition on
page~\pageref{def:bridgestability} could be adjusted to change the set of
stable bridges or focus more on address changes or on fractional uptime.
We're leaving the fine-tuning for future work when specifying and
implementing the BridgeDB extension.

Figure~\ref{fig:stablebridge} shows the fraction of stable bridges over
time.
If we only require bridges to meet or exceed the median WMTBAC or the
fixed value of 30 days, roughly 55~\% of the bridges are considered as
stable.
If bridges are only required to meet or exceed the WT and WFU values,
about $7/8 \times 1/2 = 43.75~\%$ of bridges are considered as stable.
Requiring both WFU and WMTBAC leads to a fraction of roughly 35~\% stable
bridges.

\begin{figure}[t]
\includegraphics[width=\textwidth]{stablebridge.pdf}
\caption{Impact of requiring stable bridges to meet or exceed the median
WFU and/or WMTBAC on the fraction of running bridges considered as stable}
\label{fig:stablebridge}
\end{figure}

The fraction of 33~\% stable bridges seems appropriate if 1 out of
3~bridges in the BridgeDB results is supposed to be a stable bridge.
If more than 1~bridge should be a stable bridge, the requirements need to
be lowered, so that a higher fraction of bridges is considered stable.
Otherwise, the load on stable bridges might become too high.

\subsection{Evaluating different requirements on stable bridges}
\label{sec:selectedstability}

The main purpose of this analysis is to compare the quality of certain
requirements and requirement combinations on the stability of selected
bridges.
Similar to the previous section, we only compare whether or not the WMTBAC
or WFU requirement is used, but don't change their parameters.

The first result is the future uptime that we can expect from a bridge
that we consider stable.
We calculate future uptime similar to past uptime by weighting events in
the near future more than those happening later.
We are particularly interested in the almost worst-case scenario here,
which is why we're looking at the 10th percentile weighted fractional
uptime in the future.
This number means that 10~\% of bridges have a weighted fractional uptime
at most this high and 90~\% of bridges have a value at least this high.

Figure~\ref{fig:fwfu-sim} visualizes the four possible combinations of
using or not using the WMTBAC and WFU requirements.
In this plot, the ``WFU \& WMTBAC'' and ``WFU'' lines almost entirely
overlap, meaning that the WMTBAC requirement doesn't add anything to
future uptime of selected bridges.
If the WFU requirement is not used, requiring bridges to meet the WMTBAC
requirement increases future uptime from roughly 35~\% to maybe 55~\%.
That means that there is a slight correlation between the two metrics,
which is plausible.

\begin{figure}[t]
\includegraphics[width=\textwidth]{fwfu-sim.pdf}
\caption{Impact of requiring stable bridges to meet or exceed the median
WFU and/or WMTBAC on the 10th percentile weighted fractional uptime in the
future}
\label{fig:fwfu-sim}
\end{figure}

The second result is the time that a selected bridge stays on the same
address and port.
We simply measure the time that the bridge will keep using its current
address in days.
Again, we look at the 10th percentile.
90~\% of selected bridges keep their address longer than this time.

Figure~\ref{fig:tosa-sim} shows for how long bridges keep their address
and port.
Bridges meeting both WFU and WTMBAC requirements keep their address for 2
to 5~weeks.
This value decreases to 1 to 3~weeks when taking away the WFU requirement,
which is also a result of the two metrics beeing correlated.
The bridges that only meet the WFU requirement and not the WMTBAC
requirement change their address within the first week.
If we don't use any requirement at all, which is what BridgeDB does today,
10~\% of all bridges change their address within a single day.

\begin{figure}[t]
\includegraphics[width=\textwidth]{tosa-sim.pdf}
\caption{Impact of requiring stable bridges to meet or exceed the median
WFU and/or WMTBAC on the 10th percentile time on the same address}
\label{fig:tosa-sim}
\end{figure}

\section{Concluding the bridge stability analysis}

In this report we propose to extend BridgeDB to make it give out at least
one stable bridge per user.
Bridge stability can be calculated based on bridge status information over
time, similar to how the directory authorities calculate relay stability.
The bridge stability metric proposed here is based on a bridge's past
uptime and the frequency of changing its address and/or port.
Requiring at least 1 bridge of the 3 to be given out to users greatly
reduces the worst case probability of all bridges being offline or
changing their addresses or ports.
The price for this increase in stability is that stable bridges will be
given out more often than non-stable bridges and will therefore see more
usage.

We suggest to implement the described bridge stability metric in BridgeDB
and make it configurable to tweak the requirement parameters if needed.
Maybe it turns out to be more useful to lower the requirements for a
bridge to become stable and give out two stable bridges per response.
It's also possible that the requirement for a bridge to keep its address
becomes less important in the future when bridge clients can request a
bridge's current address from the bridge authority.
All these scenarios can be analyzed before deploying them using archived
data as done in this report.

\bibliography{report}
\bibliographystyle{plain}

\end{document}

