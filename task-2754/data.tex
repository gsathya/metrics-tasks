\documentclass{article}
\usepackage{url}
\usepackage[pdftex]{graphicx}
\usepackage{graphics}
\usepackage{color}
\begin{document}
\title{Overview of Statistical Data in the Tor Network}
\author{Karsten Loesing}
\maketitle

\section{Introduction}

Statistical analysis in the Tor network can be performed using various
kinds of data.
In this report we give an overview of three major data sources for
statistics in the Tor network:
First, we recap measuring the Tor network from public directory
information \cite{loesing2009measuring} in Section~\ref{sec:serverdesc}
and explain the sanitzation process of (non-public) bridge directory
information in Section~\ref{sec:bridgesan}.
Second, we describe the numerous aggregate statistics that relays publish
about their usage \cite{loesing2010case}
in Sections~\ref{sec:bytehist} to \ref{fig:connbidirect}.
Third, we delineate the output of various Tor services like GetTor or
Tor Check as well as specific measurement tools like Torperf in
Sections~\ref{sec:torperf} to \ref{sec:exitlist}.
All data described in this report are available for download on the
metrics
website.\footnote{\texttt{https://metrics.torproject.org/data.html}}

\section{Server descriptors and network statuses}
\label{sec:serverdesc}

Relays in the Tor network report their capabilities by publishing server
descriptors to the directory authorities.
The directory authorities confirm reachability of relays and assign flags
to help clients make good path selections.
Every hour, the directory authorities publish a network status consensus
with all known running relays at the time.
Both server descriptors and network statuses constitute a solid data basis
for statistical analysis in the Tor network.
We described the approach to measure the Tor network from public directory
information in \cite{loesing2009measuring} and provide interactive
graphs on the metrics
website.\footnote{\texttt{https://metrics.torproject.org/graphs.html}}
In this section, we briefly describe the most interesting pieces of the
two descriptor formats that can be used for statistics.

\paragraph{Server descriptors}

The server descriptors published by relays at least once every 18 hours
contain the necessary information for clients to build circuits using a
given relay.
These server descriptors can also be useful for statistical analysis of
the Tor network infrastructure.

We assume that the majority of server descriptors are correct.
But when performing statistical analysis on server descriptors, one has to
keep in mind that only a small subset of the information written to server
descriptors is confirmed by the trusted directory authorities.
In theory, relays can provide false information in their server
descriptors, even though the incentive to do so is probably low.

Figure~\ref{fig:serverdesc} shows an example server descriptor.
The following data fields in server descriptors may be relevant to
statistical analysis:

\begin{figure}
\begin{verbatim}
router blutmagie 192.251.226.206 443 0 80
platform Tor 0.2.2.20-alpha on Linux x86_64
opt protocols Link 1 2 Circuit 1
published 2010-12-27 14:35:27
opt fingerprint 6297 B13A 687B 521A 59C6 BD79 188A 2501 EC03 A065
uptime 445412
bandwidth 14336000 18432000 15905178
opt extra-info-digest 5C1D5D6F8B243304079BC15CD96C7FCCB88322D4
opt caches-extra-info
onion-key
[...]
signing-key
[...]
family $66CA87E164F1CFCE8C3BB5C095217A28578B8BAF
  $67EC84376D9C4C467DCE8621AACA109160B5264E
  $7B698D327F1695590408FED95CDEE1565774D136
opt hidden-service-dir
contact abuse@blutmagie.de
reject 0.0.0.0/8:*
reject 169.254.0.0/16:*
reject 127.0.0.0/8:*
reject 192.168.0.0/16:*
reject 10.0.0.0/8:*
reject 172.16.0.0/12:*
reject 192.251.226.206:*
reject *:25
reject *:119
reject *:135-139
reject *:445
reject *:465
reject *:563
reject *:587
reject *:1214
reject *:4661-4666
reject *:6346-6429
reject *:6660-6999
accept *:*
router-signature
[...]
\end{verbatim}
\vspace{-1em}
\caption{Server descriptor published by relay \texttt{blugmagie} (without
cryptographic keys and hashes)}
\label{fig:serverdesc}
%----------------------------------------------------------------
\end{figure}

\begin{itemize}
\item \textit{IP address and ports:} Relays provide their IP address
and ports where they accept requests to build circuits and directory
requests.
These data fields are contained in the first line of a server descriptor
starting with \verb+router+.
Note that in rare cases, the IP address provided here can be different
from the IP address used for exiting to the Internet.
The latter can be found in the exit lists produced by Tor Check as
described in Section~\ref{sec:exitlist}.
\item \textit{Operating system and Tor software version:} Relays include
their operating system and Tor software version in their server
descriptors in the \verb+platform+ line.
While this information is very likely correct in most cases, a few relay
operators may try to impede hacking attempts by providing false platform
strings.
\item \textit{Uptime:} Relays include the number of seconds since the
last restart in their server descriptor in the \verb+uptime+ line.
\item \textit{Own measured bandwidth:} Relays report the bandwidth that
they are willing to provide on average and for short periods of time.
Relays also perform periodic bandwidth self-tests and report their actual
available bandwidth.
The latter was used by clients to weight relays in the path selection
algorithm and was sometimes subject to manipulation by malicious relays.
All three bandwidth values can be found in a server descriptor's
\verb+bandwidth+ line.
With the introduction of bandwidth scanners, the self-reported relay
bandwidth in server descriptors has become less
relevant.\footnote{\url{http://gitweb.torproject.org/torflow.git/}}
\item \textit{Relay family:} Some relay operators who run more than one
relay organize their relays in relay families, so that clients don't pick
more than one of these relays for a single circuit.
Each relay belonging to a relay family lists the members of that family
either by nickname or fingerprint in its server descriptor in the
\verb+family+ line.
\item \textit{Exit policy:} Relays define their exit policy by including
firewall-like rules which outgoing connections they reject or accept in
the \verb+reject+ and \verb+accept+ lines.
\end{itemize}

These are just a subset of the fields in a server descriptor that seem
relevant for statistical analysis.
For a complete list of fields in server descriptors, see the directory
procol specification \cite{dirspec}.

\paragraph{Network statuses}

Every hour, the directory authorities publish a new network status that
contains a list of all running relays.
The directory authorities confirm reachability of the contained relays and
assign flags based on the relays' characteristics.
The entries in a network status reference the last published server
descriptor of a relay.

The network statuses are relevant for statistical analysis, because they
constitute trusted snapshots of the Tor network.
Anyone can publish as many server descriptors as they want, but only the
directory authorities can confirm that a relay was running at a given
time.
Most statistics on the Tor network infrastructure rely on network statuses
and possibly combine them with the referenced server descriptors.
Figure~\ref{fig:statusentry} shows the network status entry referencing
the server descriptor from Figure~\ref{fig:serverdesc}.
In addition to the reachability information, network statuses contain the
following fields that may be relevant for statistical analysis:

\begin{figure}
\begin{verbatim}
r blutmagie YpexOmh7UhpZxr15GIolAewDoGU
  lFY7WmD/yvVFp9drmZzNeTxZ6dw 2010-12-27 14:35:27 192.251.226.206
  443 80
s Exit Fast Guard HSDir Named Running Stable V2Dir Valid
v Tor 0.2.2.20-alpha
w Bandwidth=30800
p reject 25,119,135-139,445,465,563,587,1214,4661-4666,6346-6429,
  6660-6999
\end{verbatim}
%----------------------------------------------------------------
\vspace{-1em}
\caption{Network status entry of relay \texttt{blutmagie}}
\label{fig:statusentry}
\end{figure}

\begin{itemize}
\item \textit{Relay flags:} The directory authorities assign flags to
relays based on their characteristics to the line starting with \verb+s+.
Examples are the \verb+Exit+ flag if a relay permits exiting to the
Internet and the \verb+Guard+ flag if a relay is stable enough to be
picked as guard node
\item \textit{Relay version:} The directory authorities include the
version part of the platform string written to server descriptors in the
network status in the line starting with \verb+v+.
\item \textit{Bandwidth weights:} The network status contains a bandwidth
weight for every relay in the lines with \verb+w+ that clients shall use
for weighting relays in their path selection algorithm.
This bandwidth weight is either the self-reported bandwidth of the relay
or the bandwidth measured by the bandwidth scanners.
\item \textit{Exit policy summary:} Every entry in a network status
contains a summary version of a relay's exit policy in the line starting
with \verb+p+.
This summary is a list of accepted or rejected ports for exit to most IP
addresses.
\end{itemize}

\section{Sanitized bridge descriptors}
\label{sec:bridgesan}

Bridges in the Tor network publish server descriptors to the bridge
authority which in turn generates a bridge network status.
We cannot, however, make the bridge server descriptors and bridge network
statuses available for statistical analysis as we do with the relay server
descriptors and relay network statuses.
The problem is that bridge server descriptors and network statuses contain
bridge IP addresses and other sensitive information that shall not be made
publicly available.
We therefore sanitize bridge descriptors by removing all potentially
identifying information and publish sanitized versions of the descriptors.
The processing steps for sanitizing bridge descriptors are as follows:

\begin{enumerate}
\item \textit{Replace the bridge identity with its SHA1 value:} Clients
can request a bridge's current descriptor by sending its identity string
to the bridge authority.
This is a feature to make bridges on dynamic IP addresses useful.
Therefore, the original identities (and anything that could be used to
derive them) need to be removed from the descriptors.
The bridge identity is replaced with its SHA1 hash value.
The idea is to have a consistent replacement that remains stable over
months or even years (without keeping a secret for a keyed hash function).
\item \textit{Remove all cryptographic keys and signatures:} It would be
straightforward to learn about the bridge identity from the bridge's
public key.
Replacing keys by newly generated ones seemed to be unnecessary (and would
involve keeping a state over months/years), so that all cryptographic
objects have simply been removed. 
\item \textit{Replace IP address with IP address hash:} Of course, the IP
address needs to be removed, too.
It is replaced with \verb+10.x.x.x+ with \verb+x.x.x+ being the 3 byte
output of \verb+H(IP address | bridge identity | secret)[:3]+.
The input \verb+IP address+ is the 4-byte long binary representation of
the bridge's current IP address.
The \verb+bridge identity+ is the 20-byte long binary representation of
the bridge's long-term identity fingerprint.
The \verb+secret+ is a 31-byte long secure random string that changes once
per month for all descriptors and statuses published in that month.
\verb+H()+ is SHA-256.
The \verb+[:3]+ operator means that we pick the 3 most significant bytes
of the result.
\item \textit{Replace contact information:} If there is contact
information in a descriptor, the contact line is changed to
\verb+somebody+.
\item \textit{Replace nickname with Unnamed:} The bridge nicknames might
give hints on the location of the bridge if chosen without care; e.g.\ a
bridge nickname might be very similar to the operators' relay nicknames
which might be located on adjacent IP addresses.
All bridge nicknames are therefore replaced with the string
\verb+Unnamed+.
\end{enumerate}

Figure~\ref{fig:bridgeserverdesc} shows an example bridge server
descriptor that is referenced from the bridge network status entry in
Figure~\ref{fig:bridgestatusentry}.
For more details about this process, see the bridge descriptor sanitizer
and the metrics database
software.\footnote{\texttt{https://metrics.torproject.org/tools.html}}

\begin{figure}
\begin{verbatim}
router Unnamed 10.74.150.129 443 0 0
platform Tor 0.2.2.19-alpha (git-1988927edecce4c7) on Linux i686
opt protocols Link 1 2 Circuit 1
published 2010-12-27 18:55:01
opt fingerprint A5FA 7F38 B02A 415E 72FE 614C 64A1 E5A9 2BA9 9BBD
uptime 2347112
bandwidth 5242880 10485760 1016594
opt extra-info-digest 86E6E9E68707AF586FFD09A36FAC236ADA0D11CC
opt hidden-service-dir
contact somebody
reject *:*
\end{verbatim}
\vspace{-1em}
\caption{Sanitized bridge server descriptor}
\label{fig:bridgeserverdesc}
%----------------------------------------------------------------
\end{figure}

\begin{figure}
\begin{verbatim}

r Unnamed pfp/OLAqQV5y/mFMZKHlqSupm70 dByzfWWLas9cen7PtZ3XGYIJHt4
  2010-12-27 18:55:01 10.74.150.129 443 0
s Fast Guard HSDir Running Stable Valid
\end{verbatim}
\vspace{-1em}
\caption{Sanitized bridge network status entry}
\label{fig:bridgestatusentry}
%----------------------------------------------------------------
\end{figure}

\section{Byte histories}
\label{sec:bytehist}

Relays include aggregate statistics in their descriptors that they upload
to the directory authorities.
These aggregate statistics are contained in extra-info descriptors that
are published in companion with server descriptors.
Extra-info descriptors are not required for clients to build circuits.
An extra-info descriptor belonging to a server descriptor is referenced by
its SHA1 hash value.

Byte histories were the first statistical data that relays published about
their usage.
Relays report the number of written and read bytes in 15-minute intervals
throughout the last 24 hours.
The extra-info descriptor in Figure~\ref{fig:extrainfo} contains the byte
histories in the two lines starting with \verb+write-history+ and
\verb+read-history+.
More details about these statistics can be found in the directory protocol
specification~\cite{dirspec}.

\begin{figure}
\begin{verbatim}
extra-info blutmagie 6297B13A687B521A59C6BD79188A2501EC03A065
published 2010-12-27 14:35:27
write-history 2010-12-27 14:34:05 (900 s) 12902389760,
  12902402048,12859373568,12894131200,[...]
read-history 2010-12-27 14:34:05 (900 s) 12770249728,12833485824,
  12661140480,12872439808,[...]
dirreq-write-history 2010-12-27 14:26:13 (900 s) 51731456,
  60808192,56740864,54948864,[...]
dirreq-read-history 2010-12-27 14:26:13 (900 s) 4747264,4767744,
  4511744,4752384,[...]
dirreq-stats-end 2010-12-27 10:51:09 (86400 s)
dirreq-v3-ips us=2000,de=1344,fr=744,kr=712,[...]
dirreq-v2-ips ??=8,au=8,cn=8,cz=8,[...]
dirreq-v3-reqs us=2368,de=1680,kr=1048,fr=800,[...]
dirreq-v2-reqs id=48,??=8,au=8,cn=8,[...]
dirreq-v3-resp ok=12504,not-enough-sigs=0,unavailable=0,
  not-found=0,not-modified=0,busy=128
dirreq-v2-resp ok=64,unavailable=0,not-found=8,not-modified=0,
  busy=8
dirreq-v2-share 1.03%
dirreq-v3-share 1.03%
dirreq-v3-direct-dl complete=316,timeout=4,running=0,min=4649,
  d1=36436,d2=68056,q1=76600,d3=87891,d4=131294,md=173579,
  d6=229695,d7=294528,q3=332053,d8=376301,d9=530252,max=2129698
dirreq-v2-direct-dl complete=16,timeout=52,running=0,min=9769,
  d1=9769,d2=9844,q1=9981,d3=9981,d4=27297,md=33640,d6=60814,
  d7=205884,q3=205884,d8=361137,d9=628256,max=956009
dirreq-v3-tunneled-dl complete=12088,timeout=92,running=4,
  min=534,d1=31351,d2=49166,q1=58490,d3=70774,d4=88192,md=109778,
  d6=152389,d7=203435,q3=246377,d8=323837,d9=559237,max=26601000
dirreq-v2-tunneled-dl complete=0,timeout=0,running=0
entry-stats-end 2010-12-27 10:51:09 (86400 s)
entry-ips de=11024,us=10672,ir=5936,fr=5040,[...]
exit-stats-end 2010-12-27 10:51:09 (86400 s)
exit-kibibytes-written 80=6758009,443=498987,4000=227483,
  5004=1182656,11000=22767,19371=1428809,31551=8212,41500=965584,
  51413=3772428,56424=1912605,other=175227777
exit-kibibytes-read 80=197075167,443=5954607,4000=1660990,
  5004=1808563,11000=1893893,19371=130360,31551=7588414,
  41500=756287,51413=2994144,56424=1646509,other=288412366
exit-streams-opened 80=5095484,443=359256,4000=4508,5004=22288,
  11000=124,19371=24,31551=40,41500=96,51413=16840,56424=28,
  other=1970964
\end{verbatim}
%----------------------------------------------------------------
\vspace{-1em}
\caption{Extra-info descriptor published by relay \texttt{blutmagie}
(without cryptographic signature and with long lines being truncated)}
\label{fig:extrainfo}
\end{figure}

\section{Directory requests}

The directory authorities and directory mirrors report statistical data
about processed directory requests.
Starting with Tor version 0.2.2.15-alpha, all directories report the
number of written and read bytes for answering directory requests.
The format is similar to the format of byte histories as described in the
previous section.
The relevant lines are \verb+dirreq-write-history+ and
\verb+dirreq-read-history+ in Figure~\ref{fig:extrainfo}.
These two lines contain the subset of total read and written bytes that
the directory mirror spent on responding to any kind of directory request,
including network statuses, server descriptors, extra-info descriptors,
authority certificates, etc.

The directories further report statistics on answering directory requests
for network statuses only.
For Tor versions before 0.2.3.x, relay operators had to manually enable
these statistics, which is why only a few directories report them.
The lines starting with \verb+dirreq-v3-+ all belong to the directory
request statistics (the lines starting with \verb+dirreq-v2-+ report
similar statistics for version 2 of the directory protocol which is
deprecated at the time of writing this report).
The following fields may be relevant for statistical analysis:

\begin{itemize}
\item \textit{Unique IP addresses:} The numbers in \verb+dirreq-v3-ips+
denote the unique IP addresses of clients requesting network statuses by
country.
\item \textit{Network status requests:} The numbers in
\verb+dirreq-v3-reqs+ constitute the total network status requests by
country.
\item \textit{Request share:} The percentage in \verb+dirreq-v3-share+ is
an estimate of the share of directory requests that the reporting relay
expects to see in the Tor network.
In \cite{hahn2010privacy} we found that this estimate isn't very useful
for statistical analysis because of the different approaches that clients
take to select directory mirrors.
The fraction of written directory bytes (\verb+dirreq-write-history+) can
be used to derive a better metric for the share of directory requests.
\item \textit{Network status responses:} The directories also report
whether they could provide the requested network status to clients in
\verb+dirreq-v3-resp+.
This information was mostly used to diagnose error rates in version 2 of
the directory protocol where a lot of directories replied to network
status requests with \verb+503 Busy+.
In version 3 of the directory protocol, most responses contain the status
code \verb+200 OK+.
\item \textit{Network status download times:} The line
\verb+dirreq-v3-direct-dl+ contains statistics on the download of network
statuses via the relay's directory port.
The line \verb+dirreq-v3-tunneled-dl+ contains similar statistics on
downloads via a 1-hop circuit between client and directory (which is the
common approach in version 3 of the directory protocol).
Relays report how many requests have been completed, have timed out, and
are still running at the end of a 24-hour time interval as well as the
minimum, maximum, median, quartiles, and deciles of download times.
\end{itemize}
More details about these statistics can be found in the directory protocol
specification~\cite{dirspec}.

\section{Connecting clients}

Relays can be configured to report per-country statistics on directly
connecting clients.
This metric includes clients connecting to a relay in order to build
circuits and clients creating a 1-hop circuit to request directory
information.
In practice, the latter number outweighs the former number.
The \verb+entry-ips+ line in Figure~\ref{fig:extrainfo} shows the number
of unique IP addresses connecting to the relay by country.
More details about these statistics can be found in the directory protocol
specification~\cite{dirspec}.

\section{Bridge users}

Bridges report statistics on connecting bridge clients in their extra-info
descriptors.
Figure~\ref{fig:bridgeextrainfo} shows a bridge extra-info descriptor
with the bridge user statistics in the \verb+bridge-ips+ line.

\begin{figure}
\begin{verbatim}
extra-info Unnamed A5FA7F38B02A415E72FE614C64A1E5A92BA99BBD
published 2010-12-27 18:55:01
write-history 2010-12-27 18:43:50 (900 s) 151712768,176698368,
  180030464,163150848,[...]
read-history 2010-12-27 18:43:50 (900 s) 148109312,172274688,
  172168192,161094656,[...]
bridge-stats-end 2010-12-27 14:56:29 (86400 s)
bridge-ips sa=48,us=40,de=32,ir=32,[...]
\end{verbatim}
\vspace{-1em}
\caption{Sanitized bridge extra-info descriptor}
%----------------------------------------------------------------
\label{fig:bridgeextrainfo}
\end{figure}

Bridges running Tor version 0.2.2.3-alpha or earlier report bridge users
in a similar line starting with \verb+geoip-client-origins+.
The reason for switching to \verb+bridge-ips+ was that the measurement
interval in \verb+geoip-client-origins+ had a variable length, whereas the
measurement interval in 0.2.2.4-alpha and later is set to exactly
24~hours.
In order to clearly distinguish the new measurement intervals from the old
ones, the new keywords have been introduced.
More details about these statistics can be found in the directory protocol
specification~\cite{dirspec}.

\section{Cell-queue statistics}

Relays can be configured to report aggregate statistics on their cell
queues.
These statistics include average processed cells, average number of queued
cells, and average time that cells spend in circuits.
Circuits are split into deciles based on the number of processed cells.
The statistics are provided for circuit deciles from loudest to quietest
circuits.
Figure~\ref{fig:cellstats} shows the cell statistics contained in an
extra-info descriptor by relay \texttt{gabelmoo}.
An early analysis of cell-queue statistics can be found in
\cite{loesing2009analysis}.
More details about these statistics can be found in the directory protocol
specification~\cite{dirspec}.

\begin{figure}
\begin{verbatim}
cell-stats-end 2010-12-27 09:59:50 (86400 s)
cell-processed-cells 4563,153,42,15,7,7,6,5,4,2
cell-queued-cells 9.39,0.98,0.09,0.01,0.00,0.00,0.00,0.01,0.00,
  0.01
cell-time-in-queue 2248,807,277,92,49,22,52,55,81,148
cell-circuits-per-decile 7233
\end{verbatim}
%----------------------------------------------------------------
\vspace{-1em}
\caption{Cell statistics in extra-info descriptor by relay
\texttt{gabelmoo}}
\label{fig:cellstats}
\end{figure}

\section{Exit-port statistics}

Exit relays running Tor version 0.2.1.1-alpha or higher can be configured
to report aggregate statistics on exiting connections.
These relays report the number of opened streams, written and read bytes
by exiting port.
Until version 0.2.2.19-alpha, relays reported all ports exceeding a
threshold of 0.01~\% of all written and read exit bytes.
Starting with version 0.2.2.20-alpha, relays only report the top 10 ports
in exit-port statistics in order not to exceed the maximum extra-info
descriptor length of 50 KB.
Figure~\ref{fig:extrainfo} on page \pageref{fig:extrainfo} contains
exit-port statistics in the lines starting with \verb+exit-+.
More details about these statistics can be found in the directory protocol
specification~\cite{dirspec}.

\section{Bidirectional connection use}
\label{sec:bidistats}

Relays running Tor version 0.2.3.1-alpha or higher can be configured to
report what fraction of connections is used uni- or bi-directionally.
Every 10 seconds, relays determine for every connection whether they read
and wrote less than a threshold of 20 KiB.
Connections below this threshold are labeled as ``Below Threshold''.
For the remaining connections, relays report whether they read/wrote at
least 10 times as many bytes as they wrote/read.
If so, they classify a connection as ``Mostly reading'' or ``Mostly
writing,'' respectively.
All other connections are classified as ``Both reading and writing.''
After classifying connections, read and write counters are reset for the
next 10-second interval.
Statistics are aggregated over 24 hours.
Figure~\ref{fig:connbidirect} shows the bidirectional connection use
statistics in an extra-info descriptor by relay \texttt{zweifaltigkeit}.
The four numbers denote the number of connections ``Below threshold,''
``Mostly reading,'' ``Mostly writing,'' and ``Both reading and writing.''
More details about these statistics can be found in the directory protocol
specification~\cite{dirspec}.

\begin{figure}
\begin{verbatim}
conn-bi-direct 2010-12-28 15:55:11 (86400 s) 387465,45285,55361,
  81786
\end{verbatim}
%----------------------------------------------------------------
\vspace{-1em}
\caption{Bidirectional connection use statistic in extra-info descriptor
by relay \texttt{zweifaltigkeit}}
\label{fig:connbidirect}
\end{figure}

\section{Torperf output files}
\label{sec:torperf}

Torperf is a little tool that measures Tor's performance as users
experience it.
Torperf uses a trivial SOCKS client to download files of various sizes
over the Tor network and notes how long substeps take.
Torperf can be downloaded from the metrics
website.\footnote{\texttt{https://metrics.torproject.org/tools.html}}

Torperf can produce two output files: \verb+.data+ and \verb+.extradata+.
The \verb+.data+ file contains timestamps for nine substeps and the byte
summaries for downloading a test file via Tor.
Figure~\ref{fig:torperf} shows an example output of a Torperf run.
The timestamps in the upper part of this output are seconds and
nanoseconds since 1970-01-01 00:00:00.000000.

Torperf can be configured to write \verb+.extradata+ files by attaching
a Tor controller and writing certain controller events to disk.
The content of a \verb+.extradata+ line is shown in the lower part of
Figure~\ref{fig:torperf}.
The first column indicates if this circuit was actually used to fetch
the data (\verb+ok+) or if Tor chose a different circuit because this
circuit was problematic (\verb+error+).
For every \verb+error+ entry there should be a following \verb+ok+ entry,
unless the network of the Torperf instance is dead or the resource is
unavailable.
The circuit build completion time in the \verb+.extradata+ line is the
time between Torperf sent a SOCKS request and received a SOCKS response in
the \verb+.data+ file.
The three or more hops of the circuit are listed by relay fingerprint and
nickname.
An \verb+=+ sign between the two means that a relay has the \texttt{Named}
flag, whereas the \verb+~+ sign means it doesn't.

\begin{figure}
\begin{verbatim}
# Timestamps and byte summaries contained in .data files:
1293543301 762678   # Connection process started
1293543301 762704   # After socket is created
1293543301 763074   # After socket is connected
1293543301 763190   # After authentication methods are negotiated
                    # (SOCKS 5 only)
1293543301 763816   # After SOCKS request is sent
1293543302 901783   # After SOCKS response is received
1293543302 901818   # After HTTP request is written
1293543304 445732   # After first response is received
1293543305 456664   # After payload is complete
75                  # Written bytes
51442               # Read bytes

# Path information contained in .extradata files:
ok                  # Status code
1293543302          # Circuit build completion time
$2F265B37920BDFE474BF795739978EEFA4427510=fejk4        # 1st hop
$66CA87E164F1CFCE8C3BB5C095217A28578B8BAF=blutmagie3   # 2nd hop
$76997E6557828E8E57F70FDFBD93FB3AA470C620~Amunet8      # 3rd hop
\end{verbatim}
%----------------------------------------------------------------
\vspace{-1em}
\caption{Torperf output lines for a single request to download a 50~KiB
file (reformatted and annotated with comments)}
\label{fig:torperf}
\end{figure}

\section{GetTor statistics file}

GetTor allows users to fetch the Tor software via email.
GetTor keeps internal statistics on the number of packages requested
every day and writes these statistics to a file.
Figure~\ref{fig:gettor} shows the statistics file for December 27, 2010.
The \verb+None+ entry stands for requests that don't ask for a specific
bundle, e.g.\ requests for the bundle list.

\begin{figure}
\begin{verbatim}
2010-12-27 - None:167 macosx-i386-bundle:0 macosx-ppc-bundle:0
  source-bundle:2 tor-browser-bundle:0 tor-browser-bundle_ar:0
  tor-browser-bundle_de:0 tor-browser-bundle_en:39
  tor-browser-bundle_es:0 tor-browser-bundle_fa:5
  tor-browser-bundle_fr:0 tor-browser-bundle_it:0
  tor-browser-bundle_nl:0 tor-browser-bundle_pl:0
  tor-browser-bundle_pt:0 tor-browser-bundle_ru:0
  tor-browser-bundle_zh_CN:77 tor-im-browser-bundle:0
  tor-im-browser-bundle_ar:0 tor-im-browser-bundle_de:0
  tor-im-browser-bundle_en:1 tor-im-browser-bundle_es:0
  tor-im-browser-bundle_fa:0 tor-im-browser-bundle_fr:0
  tor-im-browser-bundle_it:0 tor-im-browser-bundle_nl:0
  tor-im-browser-bundle_pl:0 tor-im-browser-bundle_pt:0
  tor-im-browser-bundle_ru:0 tor-im-browser-bundle_zh_CN:0
\end{verbatim}
%----------------------------------------------------------------
\vspace{-1em}
\caption{GetTor statistics file for December 27, 2010}
\label{fig:gettor}
\end{figure}

\section{Tor Check exit lists}
\label{sec:exitlist}

TorDNSEL is an implementation of the active testing, DNS-based exit list
for Tor exit
nodes.\footnote{\texttt{https://www.torproject.org/tordnsel/dist/}}
Tor Check makes the list of known exits and corresponding exit IP
addresses available in a specific format.
Figure~\ref{fig:exitlist} shows an entry of the exit list written on
December 28, 2010 at 15:21:44 UTC.
This entry means that the relay with fingerprint \verb+63Ba..+ which
published a descriptor at 07:35:55 and was contained in a version 2
network status from 08:10:11 uses two different IP addresses for exiting.
The first address \verb+91.102.152.236+ was found in a test performed at
07:10:30.
When looking at the corresponding server descriptor, one finds that this
is also the IP address on which the relay accepts connections from inside
the Tor network.
A second test performed at 10:35:30 reveals that the relay also uses IP
address \verb+91.102.152.227+ for exiting.

\begin{figure}
\begin{verbatim}
ExitNode 63BA28370F543D175173E414D5450590D73E22DC
Published 2010-12-28 07:35:55
LastStatus 2010-12-28 08:10:11
ExitAddress 91.102.152.236 2010-12-28 07:10:30
ExitAddress 91.102.152.227 2010-12-28 10:35:30
\end{verbatim}
\vspace{-1em}
\caption{Exit list entry written on December 28, 2010 at 15:21:44 UTC}
\label{fig:exitlist}
\end{figure}

\bibliography{data}
\bibliographystyle{plain}

\end{document}

